\section{Living Across the Border}

It is quite common to commute to work to Luxembourg from one of the bordering countries. There are a number of issues to keep in mind:
\begin{enumerate}
\item Tax issues: Due to double taxation programs, you pay all your income tax in Luxembourg and none in the country of your residence. If you have other income streams besides your salary in Luxembourg, you need to declare it. It will not be taxed, but used to asses the correct progressive income tax bracket for your salary.
\item Social security issues: As your health insurance is in Luxembourg, there are provisions to easily get access health care services in the country of your residence. Moreover, Luxembourg supports private pension plans within the framework given by the Juncker-scheme.
Please also consult the websites \url{http://www.diegrenzgaenger.lu} (in german) or \url{http://www.lesfrontaliers.lu} (in french).
\end{enumerate}

\subsection{Germany}

\subsubsection{Legal issues}

\begin{enumerate}
\item You need to request your tax card yourself. Until you do so, you will be taxed at the highest tax rate (currently 33\%). Once requested with form 164 NR D or F (\url{http://www.impotsdirects.public.lu/formulaires/fiches_d_impot/index.html}), it will arrive a couple of weeks later and can be sent to the HR department. If you happen to be newly employed at the end of the year and your tax card does not arrive in time, you can request to be paid back the overpaid tax by filing form 163 NR (\url{http://www.impotsdirects.public.lu/formulaires/decompte_annuel/index.html})
\item You will be insured through the Luxembourgian social security system. You can request German health insurance by sending in form S1 to a German insurance company. Once you do so, you will be issued a German healthcare identity card and can visit doctors both in Luxembourg and Germany. Usually, the form will be sent to you automatically once the university registers you as an employee.
Important: If you have an additional income with social security within Germany, you might no longer be insured in Luxembourg, but in Germany – and Germany's social security deductions are twice as high as Luxembourg’s (also be careful to check your work contract if it allows for an additional occupation).
You can own companies (e.g. be a Gesellschafter) even if the business is headquartered in Germany, as well as receive profits. But you cannot be employed in Germany and receive a salary, even if it is your own company, without your social security being subjected to German rules.
\end{enumerate}

\subsubsection{Commuting}
If you chose to include the public transport option you pay ca. 25 Euro extra at you inscription. You get a sticker on your student card and that includes all of Luxembourg's public transport (in the whole country!). Unfortunately, it does not cover any line extending outside of Luxembourg.
A popular commuting line to the Weicker building and Kirchberg campus of the SnT is bus 118 (from Trier). A monthly ticket costs 85 Euros and can be purchased in the bus with the bus driver directly.

