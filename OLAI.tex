\section{CEDIES \& OLAI (can be for Europeans only)}
%Please check, can be for Europeans only.

\subsection{Scholarship from CEDIES} 
PhD students from the EU or with a spouse or partner
from the EU (even those receiving a salary or AFR grant) can obtain a
study grant from the Luxembourgish government (also known as 'free
money'). To receive the grant, you will need to deal with some
bureaucracy, but the money is definitely worth the effort.
See \url{http://www.cedies.public.lu/fr/aides-financieres/pret-bourses/index.html}
for more information. 
WARNING: Currently, there are some changes that will not allow PhD students to apply for the grant, 
but only some loans. 
This part will be updated once the new rules of CEDIES take effect in the new academic year. 
 
Please check for yourself! The approach could look like this:
- Download the form at \url{www.cedies.public.lu}
- Get a bank account in Luxembourg (if you don't have one yet). This
is free at most banks for students, for example at BCEE (Spuerkees).
Go with your passport or national identity card, student card and
proof of residence from the municipality (``attestation
d'enregistrement'') to a bank office and ask for a student account. You
will get your bank account number immediately.
- Fill out the form you received from CEDIES. Under 'curriculum de vos
�tudes sup�rieures', fill in 'Oui' under 'Annee reusie', and fill in
nothing under ECTS.
- Join to the form the following:
1. Copy of ID card / passport
2. Certificat d'affiliation from CCSS; to be requested online at
\url{http://www.ccss.lu/certificats/assures/certificat-daffiliation/}
3. Certificate of inscription (this is one of the small grayish cards
you receive from the university after paying inscription fees)
4. Certificat de r�sidence; obtainable at your municipality for 2 euro.
If you live in Luxembourg city, this is at Hamilius bus station.
If you live in Luxembourg city, you can request the certificate also
online at \url{https://service.vdl.lu/be/} .
If you want to save 2 euro, you can also try it with a copy of the
"attestation d'enregistrement" from the municipality (please let me know
if this suffices). 
Moreover, "the certificate de menage" should be required at the municipality together with the residence certificate (so two documents to be asked at the municipality). 
5. Copies of 'fiche de salaire' of last 3 months (sent to your home
every month by the university).
6. Copy of 'Relev� d'identit� bancaire', obtainable at your bank.
 
You will receive a letter stating the amount of your grant (bourse) and
loan (pr�t). The former will be paid to your bank account in a few
weeks; the latter you probably will want to ignore.
Tip is by
-- Matthijs


\subsection{Welcome and Integration Contract (OLAI)} 
This could be a good way to help yourself to integrate easier in Luxembourg.
At least you may want to try it in order to take some language courses almost for free!.

\url{http://www.olai.public.lu/en/accueil-integration/mesures/contrat-accueil/index.html}
