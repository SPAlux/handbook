\section{Buying a car}
Buying a new car is easy, just go to a garage!

\subsection{Buying a second-hand car}
Buying a second-hand car is more tricky and you may need some more information. 
 
There are two kinds of second-hand cars: offered by the garages, or offered by privates.
You can visit websites of big garages in Luxembourg to find new cars and second-hand cars (called occassions).
For example, you can google to find garages in Luxembourg, like garage Losch.
You can also come to the garage to take a look at the occasions.
 
Of course, occasions at garages are still quite new and expensive. 
Otherwise, for a cheaper solution, you can find second-hand cars offered by privates, in Luxembourg (or France or Germany also possible).
 
 \begin{itemize}
 \item \href{http://www.luxauto.lu/fr/}{luxauto.lu}

\item %Or you can take a look also at this website:
\href{http://en.autofinder.lu/}{autofinder.lu}
 
 \item
\href{http://www.guichet.public.lu/citoyens/fr/transports-mobilite/transports-individuels/vehicule-motorise/acheter-vehicule-luxembourg/index.html}{Acheter un v�hicule immatricul� en dernier lieu au Luxembourg}
\end{itemize}
 
From Germany:\\ 
 \begin{itemize}
\item \href{http://www.mobile.de/}{www.mobile.de}

\item \href{http://www.autoscout24.de/}{www.autoscout24.de}
\end{itemize}
 

Some tips for buying a car :
 \begin{itemize}
\item
\href{http://www.wort.lu/en/view/autofestival-2013-a-buyer-s-guide-50fe5928e4b05b4a370a7527}{Autofestival 2013 - A buyer's guide}

\item
\href{http://www.wort.lu/en/view/autofestival-2013-what-to-watch-out-for-when-buying-a-second-hand-car-50fe8be8e4b080a4ac3ed05f}{Autofestival 2013 - What to watch out for when buying a second-hand car}

\item
\href{http://www.wort.lu/en/view/autofestival-2013-selling-a-used-car-step-by-step-510005b8e4b04b787264cc67}{Autofestival 2013 - Selling a used car step-by-step}
 \end{itemize}

Car loans

 \begin{itemize}
\item \href{https://www.bil.com/en/individuals/car/i-buying-a-car/finance/Pages/loan-personal.aspx}{Car loans}
 \end{itemize}
 
\subsection{Importing your car to Luxembourg}
Importing a vehicle from an European country to Luxembourg requires the following steps in order to make the process straightforward.

\begin{itemize}
	\item Obtain a copy of your residence permit from your local commune (This is only required if the vehicle you are importing was already at your name in the exiting country)
	\item Request a number plate from the Transport Ministry in Sandweiler - SNCT via email at \href{mailto:nplaques@snct.lu}{nplaques@snct.lu} (They will just ask for your social security number)
	\item Once you have a registration number you can go to the one of the following centers to print your plates:
Gr�n Signalisation - \url{http://www.grun.lu} or Lux Signalisation - \url{http://www.luxsignalisation.lu/}. Make sure you also buy a safety triangle and vest if you don�t have them already otherwise you will fail your technical inspection.
	\item With the number plate, the car and the previous immatriculations papers you can go to an insurance company to get your insurance in Luxembourg (green paper)
	\item Go the Centre Douanier in Gasperich and obtain a �Vignette 705�, make sure to bring
		\subitem The original vehicle receipt (if less than six months old or less than 6,000km) 
		\subitem Sale documents ( If the vehicle was not yours in the exiting country)
		\subitem Residence permit ( See first point )
		\subitem Car immatriculation documents from the exiting country
		\subitem The car you plan on importing (They might want to inspect it on the spot)
	\item With all this material you can go the the Technical Inspection (Contr�l Technique) in Sandweiler where at first you will be asked to hand in all the papers in the Administration area (make sure to bring at least 50 Euro cash for the immatriculation of the vehicle) and in a following step to pass the technical inspection of the vehicle itself (The queue for this can get extremely long so plan half a day just for this step).
\end{itemize}

If your technical inspection was successful you will receive all the car documents at the end of the inspection itself.
Note : This procedure might change in the future since the SNCT are currently renovating the inspection process in order to have less wait time. For further useful information :
\href{http://www.euraxess.lu/euraxess/Daily-life/Driving/Importing-your-own-car}{Importing your own car}