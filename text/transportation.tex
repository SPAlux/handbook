\section{Transportation in Luxembourg}

\subsection{Moving with Public transports}
For moving with public transport in the city and country of Luxembourg, you may use the website: \href{http://mobiliteit.lu}{mobiliteit.lu} - \href{http://travelplanner.mobiliteit.lu/hafas/cdt/query.exe/en?}{PLANIFIER un parcours}.

\subsection{Moving with a bike}

\subsubsection{Rental Bike/ Bike sharing: }
There is a city bike provider called velo�h \url{http://www.en.veloh.lu/}.
You have to register once, which is possible at every bike rent station but it turns out to not work well sometimes. Or you register on the website, print out your inscription and send it to the city of Luxembourg.
You pay once a fee of 15 Euro per year and then you can take the bikes always for half an hour for free.
Afterwards you pay one Euro per hour.
 
Grund velo have nice bikes but more expensive than city system.
 
\subsubsection{Buy a bike:}
 
Once a year at spring time there is a fair organised by Cactus Belle Etoile. It starts at the weekend when people bring bikes for sell and last whole week until the next weekend. Plenty of bikes in different price ranges with the possibility to take a ride on the place. Good way to find something cheap if you go early.
 
\href{http://Luxbazar.lu}{Luxbazar.lu}. Search for velo.
 
In Thionville there is a closest Decathlon - popular sport supermarket - but you will need to find a friend with a car.
 
\subsubsection{Fix a bike:}
The cheapest way is to go to the rental point at Grund. They fix bikes fast and often for free, symbolic price.