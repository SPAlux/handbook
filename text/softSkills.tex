\section{Some tips for developing your soft skills}

\subsection{Some useful links on how to get/finish a PhD and more}
A book on how to get a PhD: A handbook for students and their supervisors\\
\href{http://slpramod.weebly.com/uploads/3/3/2/7/3327933/how_to_get_a_phd.pdf}{How to get a
PhD}

A presentation on how to finish a PhD: Advice for Finishing that Damn Ph.D. By Prof. Daniel M. Berry (\href{mailto:dberry@abuwaterloo.ca}{dberry@abuwaterloo.ca})
UCLA, USA, 
Technion, Israel, 
University of Waterloo, Canada\\
\href{https://cs.uwaterloo.ca/~dberry/FTP_SITE/lecture.slides/finishing.phd.talk.pdf}{Advice for Finishing that Damn Ph.D.}

Some more advices for PhD students:\\
\href{https://sites.google.com/site/asergrp/advice}{Advice for Researchers and Students,
by Tao Xie }\\

%TODO: Daniel Berry, Tao Xie, etc. 
\textbf{Good Scientific Practice/Ethics in Research:}\\ \url{http://wwwen.uni.lu/content/download/55604/657605/file/University\%20of\%20Luxembourg\%20Policy\%20on\%20Ethics\%20in\%20Research_102012.pdf}\\

For more documents from the University: Research brochure, Doctoral studies, Foreign researchers' guide, Guiding principles for the valorisation of research results and intellectual property rights, Human Resources Strategy for Researchers. \\
\url{http://wwwen.uni.lu/research/downloads2}


\subsection{Courses offered by the Uni}

\subsubsection{Language courses}
Please pay attention to emails sent from the Uni, like this:

\begin{quotation}
The enrolment / re-enrolment period for the registration in the language courses for the winter semester 2013 � 2014 is now opened.
 
To proceed with your enrolment or re-enrolment in English, German, French or Luxembourgish, please click on the link below:
 
\url{https://intranet.uni.lux/index.php/intradmin/Language-courses/Offer-staff-and-doctoral-students-WS-2013-2014}
 
MANDATORY: PLEASE USE CUSTOMER CODE: � 21920P � to fill in the placement test (first registration).
 
You will be duly informed about the language courses opened a few days before the beginning of the courses.
 
For further information, do not hesitate to contact me anytime on my direct extension.
Emmanuelle Ambroisien,
Tel. : \href{callto:+352 46 66 44 9269}{+352 46 66 44 9269} (Mondays and Fridays)
 
Universit� du Luxembourg
Facult� des Lettres, des Sciences Humaines,
des Arts et des Sciences de l�Education
Route de Diekirch
L-7201 Walferdange

\end{quotation}

\subsubsection{Transferable Skills for PhD candidates}

\begin{quotation}
The Research Office together with the Doctoral Schools have organized a series of courses in transferable skills 2013.
The dates and course descriptions for the next courses can be found in the attached Course Guide, on moodle, or the Intranet (\href{https://intranet.uni.lux/index.php/intradmin/Documents-communs/Recherche/Doctoral-Education}{Doctoral Education}).
Eligible participants
In general, the trainings are open to all doctoral candidates enrolled at the University of Luxembourg.
In case of oversubscription, candidates enrolled in a doctoral school have priority if the training is part of their personal training programme.
Registration
Please register for the courses via \href{http://moodle.flshase.uni.lu/course/category.php?id=482}{FLSHASE moodle}.
Log in to moodle for making your registration. (The login to the FLSHASE moodle is possible for all PhD candidates inscribed at the UL.)
For �@uni.lu� addresses: Username = firstname.lastname
For �@student.uni.lu� addresses: Username = the student ID code
In case you have problems with the moodle login, you can contact Shahed Parnian (\href{mailto:shahed.parnian@uni.lu}{shahed.parnian@uni.lu}).
Registration opens on Friday 8th of March at 9:30am.
Further information on the registration process can be found directly on moodle or in the attached Course Guide.
Kind regards,

Daniela Schalke
Research Service
UNIVERSITY OF LUXEMBOURG
\href{http://goo.gl/Q5FYnm}{162 a, avenue de la Fa�encerie
L-1511 Luxembourg}
Office : Campus Limpertsberg, BC 1.07b
T : \href{callto:+ 352 46 66 44 9579}{+ 352 46 66 44 9579}
F : \href{+ 352 46 66 44 6215}{+ 352 46 66 44 6215}
\end{quotation}

\subsection{Toastmasters clubs and language clubs in Luxembourg}
Want to improve your public speaking and leadership skills? join a Toastmasters club would bring to you even more than you could expect! 
For more details about the Toastmasters clubs in Luxembourg, check out this link. 
\url{http://tmclub.eu/orgdata.php}
You don�t have to become a member immediately but you can visit their meetings as guests to see how they conduct the meetings and decide later if you want to join or not. 

Similarly, besides the language lunch clubs at SnT (see the next sub-section), there are some other language clubs that you can join here in Luxembourg. 
For example, you can meet people to practice speaking French every Monday at 6 pm at Independent Caf\'e (\href{http://goo.gl/aGptRc}{6 Boulevard F-d Roosevelt, 2450 Luxembourg}). 
It is for everyone who wants to improve their French, learn French or just hang out and speak French. \\
Source: \href{https://www.couchsurfing.org/n/repeating-events/next-french-meeting-luxembourg-repeats}{couchsurfing}\\
Or, you may want to take part in a language exchange in Luxembourg. ``Tandem de langues'', is a Dudelange-based initiative aimed at promoting discussion between different nationalities and communities in Dudelange by sharing language learning. \\
Source: \href{http://www.wort.lu/en/view/take-part-in-a-language-exchange-in-luxembourg-526e85e3e4b0b582f958e50f}{Take part in a language exchange in Luxembourg}

\subsection{Why not initialise your language club at SnT?}
\label{lbl_langlunch}
SPA can help you to join/set up a language club here at SnT. 
For the moment, we target the official languages used in Luxembourg first, i.e. French, German, and Luxembougish. 
We initially match a group of people that are interested in learning a language with a proficient speaker of this language. The meeting can then take place during the lunch break or after work and can be designed in a way that the group agrees upon (conversation around the table, going to a  movie in this language, etc.).
The main idea is to provide some chance for you (language learner) to practice the verbal communication in the language you are trying to master. Following a language course offered by the uni once a week may be not enough so this idea is to top up on your language course. 

If you are looking for this kind of language club, please email the SPA's representatives to express your need. 
If you are a proficient speaker of one of the official languages in Luxembourg (French, German, and Luxembougish) and would like to help others, please subscribe to run a language club by emailing the SPA's representatives. 
We would like to have 2 or more proficient speakers for each club. 
\\LATEST NEWS: Currently we have set up 2 language clubs: one learning French and another one learning German. 
Both clubs are conducted with help of native speakers. 
The French-learning group has scheduled weekly meeting on Thursday at lunch time, from 12h15. 
The German-learning group has scheduled weekly meeting on Wednesday at lunch time, from 12h15. 
For more information, please contact the SPA's representatives. 
